\chapter*{Introduction}
\addcontentsline{toc}{chapter}{Introduction}

Covalent Organic Frameworks (COFs) are a novel class of self-assembled small organic molecules, first synthesized by C\^{o}t\'e et al. in 2005\cite{cote_porous_2005}. The monomers involved in their formations are usually quite rigid due to aromaticity and condense through reversible reactions by forming covalent bonds, which leads to the formation of crystalline nanoporous networks\cite{wu_applications_2017}\cite{Kim2019} and hence homogeneous materials. This leads to materials having low density, high porosity\cite{cote_porous_2005}, and optical properties and are therefore excellent candidates for gas separation and storage\cite{wu_applications_2017}\cite{fan_covalent_2018}, electrical devices\cite{pachfule_porous-organic-framework-templated_2013}\cite{liu_hollow_2013}\cite{kim_covalent_2018}, heterogeneous catalysis\cite{bhadra_triazine_2019}, or light-harvesting\cite{dogru_photoconductive_2013}. The wide variety of applications of these materials led to an explosion of chemical and structural diversity. The Computational-Ready Experimental COF database (CoRE COF), from which the COFs used here are from, was containing 187 COF in 2017\cite{tong_exploring_2017} and 280 in 2018 \cite{tong_computation-ready_2018}.


%Indeed , first COFs were two-dimensional polymers formed by a dehydration condensation reaction of boronic acid ($RB(OH)_2$) into boroxin rings ($R_3(BO)_3$), but now, all type of linker, condensation reactions and structures have appeared like vertical linkers to induce three-dimensional structures; although the latest has not yet found as many applications as 2D ones.

% all the while being easy to functionalize both structurally and chemically,

%Their high crystallinity, compare to most polymers, yield a homogeneous material and hence reliable properties.
%Although 2D COFs are, as of now, the most promising, one main issue still remains: defining their exact crystaline structure. Unlike 3D ones, they can addopt a 
The large variety of COFs available induce the need to find way to estimate its properties in order to compare and optimise their performance for any of the applications mentioned above. To do so, the first step is to establish the exact crystalline structure of the material. Indeed, their properties are strongly dependant on their stacking structure; notably their porosity or electronic properties \cite{xu_dependence_2016} necessary for photo-electronic processes or energy storage. Although this would be straightforward for 3D-COFs because of the low degrees of freedom, such is not the case with 2D-COFs that can adopt a variety of stacking modes. But it is difficult to experimentally assess this property, among other because the amount of material to test induces costs in terms of time, labour and resources, although techniques like Nitrogen isotherms or High-Resolution Transmission Electron Microscopy have been successful in some cases. This limitation would strongly impair the development of COFs. Computational chemistry can hence play a crucial role by proposing a cheap and fast way to estimate the structure of a COF. How to use computational chemistry to properly and efficiently tackle this problem will be the matter of interest here.

%The first solution is to use experimental chemistry but the current protocols and experimental techniques are extremely time-consuming. Considering the number of material that would need to be assessed for any of the applications mentioned above, the total cost in terms of time, labor and resources would limit drastically the potential of COFs. 



%The second solution is to use computational techniques to assess quickly and cheaply a great number of materials.

%Since experimental chemistry would be unable to assess such amounts of structures, in part due to technical limitations like the high sensitivity to defects of x-ray diffraction, and the low crystallinity of these materials. In some cases, High-Resolution Transmission Electron Microscopy was able to give some property like the single layer vector but is unable to asses its stacking. Hence, the advantage of using computational chemistry becomes obvious: it makes it possible to assess the structural property like stacking, in a systematic fashion among a very large number of materials without the need for synthesis. These results can then be used to computationally estimate other properties like density, pore-volume, and even band-gap.

A naive approach to establish the stacking mode would be to start from a given structure and apply a geometry and cell optimization using Density Functional Therory (DFT). This method would not only be expensive in terms of computational costs but bear some severe limitations. Indeed, some COFs' Potential Energy Surface (PES) presents one or more local minimums, that could prevent the simulation from reaching the absolute minima.
The protocol chosen in this work is to establish the full PES for the x,y and z offset from one layer to the next. 
To achieve reasonable accuracy a great number of grid-points needs to be computed which can be very costly if performed with advanced techniques especially when the number of grid-points is consider alongside the amount of materials to test. On the other hand, if the technique used is too inacurate it might miss effects that are determining in the stacking mode. To assess the best compromise between computational cost and precision, several techniques were employed on a diverse set of COFs and their accuracy were compared. These techniques range from classical methods(Lennard-Jones and Coulombic interactions) to Tight-Binding approximation(xTB and DFTB+). Finally, a DFT optimisation was performed to serve as reference to evaluate the performance of the different methods.


%the first step of the calculation is to compute it as accurately as possible to then converge to an exact structure and obtain more complex properties of the material. The problem is now that a fine layer-to-layer x,y,z offset grid is necessary to achieve reasonable accuracy, and accurate calculations are very expensive machine-time-wise, especially when the number of grid-points is put together with the number of materials to test. To alleviate this cost, the solution chosen here is to make a first estimation of the structure using a method as cheap as possible and still gives a useful starting point for more in-depth optimization. In this setting, the work detailed below aims at finding the best compromise between calculations too expensive to be interesting as a pre-screener and the ones not accurate enough to give a useful initial guess for the rest of the treatment. With this idea in mind, different computational technics were tested on a diverse set of COFs to evaluate the complexity of calculations needed to achieve reasonable precision. These technics, from simplest to most elaborate are : the Lennard-Jones potential, Lennard-Jones and Coulombic interactions, DFTB+ \cite{aradi_dftb+_2007} and GFN2-xTB \cite{grimme_robust_2017}\cite{bannwarth_gfn2-xtb_2019}. Finally, the reference optimal structure was obtained obtained using DFT.\cite{hohenberg_inhomogeneous_1964}\cite{kohn_self-consistent_1965}

%Furthermore, by scanning all possible x,y,z offsets from one layer to the next it is possible to asses the crystallinity of the material: a steep, smooth potential would yield high crystallinity while irregularities and flatness in the Potential Energy Surface would yield lower crystallinity. This can be of the utmost importance when aiming at applications like light harvesting where irregularity in the crystalline structure can strongly impair the material's efficiency.
